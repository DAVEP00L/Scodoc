%% Original : Cléo BARAS
%%
%% Version 1.0
%%
%% Ce fichier est distribué par ScoDoc
%%
%% Si vous voulez le modifier sur le serveur, copiez le dans
%% ../local/
%% et éditez votre copie dans local.
%%

\definecolor{ugagray}{rgb}{0.42,0.42,0.42}

% define global counters
\newcounter{yearcount}


\newcounter{leftcount}

% env parcourstimeline (d'après copyright (c) 2016 Jan Küster)
% param 1: année de début 
% param 2: année de fin
% param 3: nombre de semestre
% param 4: largeur totale
\newenvironment{parcourstimeline}[4]{

	% param 1:	no
	% param 2:	Nom
	\newcommand{\parcoursevent}[3] {
		\pgfmathparse{\largeur*(##1-1)+\largeur/2+ \marge}\let\startexp\pgfmathresult

	\node[draw,color=white,] (Ev##1) at (\startexp,0.7){
			\textcolor{black}{##2}
		};
		\node[draw,color=white,] (Ev##1) at (\startexp,-0.6){
			\textcolor{black}{##3}
		};
}

	%------------------------------------------------------------
	%	BEGIN
	%------------------------------------------------------------

	\begin{tikzpicture}

	%calc pas = number of years
 	\pgfmathparse{(#4/#3)}\let\pas\pgfmathresult
			\pgfmathparse{\pas*0.9}\let\largeur\pgfmathresult
	\pgfmathparse{(#4-\largeur*#3)/2}\let\marge\pgfmathresult
	\draw[draw=ugagray,line width=2pt] (0,0) -- (#4,0) ;	%the timeline

	%for each year put a horizontal line in place
	\setcounter{yearcount}{1}
	\whiledo{\value{yearcount} < #3}{ % \fullrange}{
		\draw[fill=white,draw=ugagray, line width=2pt]  (\largeur*\value{yearcount}+\marge+0.,0) circle (0.1);
		\stepcounter{yearcount}
	}

	%start year
	\node[draw,color=ugagray,rectangle,line width=2pt,fill=ugagray!20,rounded corners=3pt] (Yd) at(0,0){\textcolor{black}{\textbf{\small#1}}};

	%end year
	\node[draw,color=ugagray,rectangle,line width=2pt,fill=ugagray!20,rounded corners=3pt] (Yd) at(#4,0){\textcolor{black}{\textbf{\small#2}}};
	

}%end begin part of newenv
{\end{tikzpicture}}